\documentclass[11pt, a4paper]{report}
%\documentclass[11pt, a4paper]{article}

%====================== PACKAGES ======================


\usepackage[french]{babel}

\frenchbsetup{StandardLists=true}
\usepackage{enumitem}
\usepackage{pifont}

\usepackage[utf8x]{inputenc}
%\usepackage[latin1]{inputenc} % Feynman ?

%pour gérer les positionnement d'images
\usepackage{float}
\usepackage{amsmath}
\DeclareMathOperator{\dt}{dt}
\usepackage{graphicx}
\usepackage{tabularx}
\usepackage[colorinlistoftodos]{todonotes}
\usepackage{url}
%pour les informations sur un document compilé en PDF et les liens externes / internes
\usepackage[pdfborder=0]{hyperref}
\hypersetup{
	colorlinks = true
	}
%pour la mise en page des tableaux
\usepackage{array}
\usepackage{tabularx}
\usepackage{multirow}
\usepackage{multicol}
\setlength{\columnsep}{50pt}
%pour utiliser \floatbarrier
%\usepackage{placeins}
%\usepackage{floatrow}
%espacement entre les lignes
\usepackage{setspace}
%modifier la mise en page de l'abstract
\usepackage{abstract}
%police et mise en page (marges) du document
\usepackage[T1]{fontenc}
\usepackage[top=2cm, bottom=2cm, left=2cm, right=2cm]{geometry}
%Pour les galerie d'images
\usepackage{subfig}

\usepackage{pdfpages}
\usepackage{tikz}
%\usepackage{tikz}
\usetikzlibrary{trees}
\usetikzlibrary{decorations.pathmorphing}
\usetikzlibrary{decorations.markings}
\usetikzlibrary{decorations.pathreplacing,calligraphy}
%\usetikzlibrary{decorations}
\usetikzlibrary{angles, quotes}
\usepackage{verbatim}

\usepackage{appendix}

\usepackage{comment}

\usepackage{xcolor}


%\PreviewEnvironment{tikzpicture}
%\setlength\PreviewBorder{0pt}%

%====================== INFORMATION ET REGLES ======================

%rajouter les numérotation pour les \paragraphe et \subparagraphe
\setcounter{secnumdepth}{4}
\setcounter{tocdepth}{4}

\hypersetup{							% Information sur le document
pdfauthor = {Stephan Runigo},			% Auteurs
pdftitle = {Documentation},			% Titre du document
pdfsubject = {Documentation},		% Sujet
pdfkeywords = {Document},	% Mots-clefs
pdfstartview={FitH}}	% ajuste la page à la largeur de l'écran
%pdfcreator = {MikTeX},% Logiciel qui a crée le document
%pdfproducer = {} % Société avec produit le logiciel
%======================== DEBUT DU DOCUMENT ========================
%
\begin{document}
%
%régler l'espacement entre les lignes
\newcommand{\HRule}{\rule{\linewidth}{0.5mm}}
%
% Titre, résumé, ... %
%
%
\begin{titlepage}
%
~\\[1cm]

\begin{center}
%\includegraphics[scale=0.5]{./presentation/chambreABulle}
\end{center}

\textsc{\Large }\\[0.5cm]

% Title \\[0.4cm]
\HRule

\begin{center}
{\huge \bfseries  Titre 1\\
titre 2\\[0.4cm] }
\end{center}

\HRule \\[1.5cm]

\begin{center}
%\includegraphics[scale=0.3]{./presentation/ptoleme}
\end{center}

\begin{center}
%\includegraphics[scale=0.3]{./presentation/diagrammesInteractions}
\end{center}


% Author and supervisor
\begin{minipage}{0.4\textwidth}
\begin{flushleft} \large
%\emph{Auteur:}\\
%Stephan \textsc{Runigo}
\end{flushleft}
\end{minipage}
\begin{minipage}{0.4\textwidth}
\begin{flushright} \large
\emph{Auteur:}\\
Stephan \textsc{Runigo}
\end{flushright}
\end{minipage}

\vfill

% Bottom of the page
{\large \today}

\end{titlepage}

\newpage
\begin{center}
\Large
Résumé
\normalsize
\end{center}
\vspace{3cm}
\begin{itemize}[leftmargin=1cm, label=\ding{32}, itemsep=21pt]
\item {\bf Objet : } .
\item {\bf Contenu : } .
\item {\bf Public concerné : } .
\end{itemize}

\vspace{3cm}

résumé.

\vspace{3cm}


%

%
% Table des matières
\tableofcontents
\thispagestyle{empty}
\setcounter{page}{0}
%
%espacement entre les lignes des tableaux
\renewcommand{\arraystretch}{1.5}
%
%====================== INCLUSION DES PARTIES ======================
%
~
\thispagestyle{empty}
%recommencer la numérotation des pages à "1"
\setcounter{page}{0}
\newpage
%
%\input{./histoire/histoire.tex}
%
%
%%%%%%%%%%%%%%%%%%%%%
\chapter{Première annexe}
%%%%%%%%%%%%%%%%%%%%%

%%%%%%%%%%%%%%%%%%%%%%%%%
\section{Premiere section}
%%%%%%%%%%%%%%%%%%%%%%%%%
%
\subsection{Première sous section}

Texte

%%%%%%%%%%%%%%%%%%%%%%%%%%%%%%%%%%%%%%%%%%%%%%%%%%%%%%%%%%%%%%%%%%%%%%%%%%%%%%%%%%%%%

%
%\input{./champs/champs.tex}
%
\input{./relativite/relativite.tex}
%
%\input{./quantique/quantique.tex}
%
%\input{./causalite/causalite.tex}
%
%
\begin{appendix}
%
%\input{./annexe/referentiel.tex}
%
\newpage
%

%%%%%%%%%%%%%%%%%%%%%
\chapter{Première annexe}
%%%%%%%%%%%%%%%%%%%%%

%%%%%%%%%%%%%%%%%%%%%%%%%
\section{Premiere section}
%%%%%%%%%%%%%%%%%%%%%%%%%
%
\subsection{Première sous section}

Texte

%%%%%%%%%%%%%%%%%%%%%%%%%%%%%%%%%%%%%%%%%%%%%%%%%%%%%%%%%%%%%%%%%%%%%%%%%%%%%%%%%%%%%

%
\newpage
%
\end{appendix}
%

%
%
%====================== INCLUSION DE LA BIBLIOGRAPHIE ======================
%
%récupérer les citation avec "/footnotemark" : 
\nocite{*}
%
% choix du style de la biblio
\bibliographystyle{plain}
%
% inclusion de la biblio
\cleardoublepage
\addcontentsline{toc}{chapter}{Bibliographie}
\bibliography{bibliographie.bib}
%
\end{document}
%%%%%%%%%%%%%%%%%%%%%%%%%%%%%%%%%%%%%%%%%%%%%%%%%%%%%%%%%%%%%%%%%%%%%%%%%%%%%%%%%
