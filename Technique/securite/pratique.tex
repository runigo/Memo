
%%%%%%%%%%%%%%%%%%%%%
\section{En pratique}
%%%%%%%%%%%%%%%%%%%%%

En pratique, 



%\section{Sécurité-confidentialité}


\section{Sécurité-conservation}

Pour ne pas perdre certaines données, il faut les décentraliser. La meilleur façon de ne pas perdre certaines données, c'est de les rendre publiques (nos photos sur notre {\it résau-social} seront toujours disponnible par l'intermédiaire de nos {\it amis}).

\section{Sécurité-confidentialité}

Pour cacher certaine données, il faut les rendre inaccessible.

Soit elle sont sur internet, alors protégées par un mot de passe, soit elle sont sur un support mobile (agenda papier, clé usb, disque dur externe)



\subsection{Mots de passe}

On peut écrire ses mots de passe dans son agenda (p. ex. mon site cloud, facebook, ...). Si on perd notre agenda il faudra songer à une procédure rapide de récupération de ses mots de passe : on va sur le site, on clique sur "mot de passe perdu", le site nous transmet un lien de récupération (il faut donc ne pas avoir perdu son mot de passe de courriel !!! que l'on devrait connaître par c{\oe}ur...), et on met un nouveau mot de passe que l'on écrit dans son nouvel agenda.

On peut écrire ses mots de passe dans un fichier, conservé dans une clé usb ou dans son PC.


\subsection{Sous-section 1} \label{labelLivre1}

\begin{itemize}[leftmargin=1cm, label=\ding{32}, itemsep=1pt]
\item {\footnotesize \bf mot} : {\it origine}, « guillemet ».
\end{itemize}


%%%%%%%%%%%%%%%%%%%%%%%%%%%%%%%%%%%%%%%%%%%%%%%%%%%%%%%%%%%%%%%%%%%%%%%%%%%
